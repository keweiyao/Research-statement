\documentclass[12pt,a4paper]{revtex4-1}
\usepackage[a4paper,total={6in, 10in}]{geometry}
\usepackage[utf8]{inputenc}
\usepackage{amsmath}
\usepackage{amsfonts}
\usepackage{amssymb}
\usepackage{hyperref}
\linespread{1.2}
\usepackage[font={footnotesize}]{caption}
\usepackage{amsthm}
\usepackage{amscd}
\usepackage{graphicx}%
\usepackage{fancyhdr}
\graphicspath{{figs/}}
\usepackage{subcaption}
\textwidth6in

\setlength{\topmargin}{0in} \addtolength{\topmargin}{-\headheight}
\addtolength{\topmargin}{-\headsep}

\setlength{\oddsidemargin}{0in}

\oddsidemargin  0.0in \evensidemargin 0.0in \parindent0em

\pagestyle{fancy}
\lhead{Research Statement} 
\rhead{\today}
\chead{{\large{\bf Weiyao Ke}}} 
\lfoot{} 
\rfoot{\bf \thepage} 
\cfoot{}

\begin{document}
I focus on the study of relativistic heavy-ion collisions and the quark-gluon plasma (QGP) using the tools of open heavy flavors.
My expertise includes computer model simulations of the medium dynamics in heavy-ion collisions, partonic transport inside a QGP using Monte-Carlo simulation techniques, and the application of Bayesian statistics and machine learning to the parameter extraction of complex models.
I shall first summarize my major research projects and future prospects. The statement also provides a complete list of research accomplishments with detailed descriptions.


\section*{Highlights of past and current research projects}
I conduct my graduate research at Duke University with Professor Steffen Bass since 2014.
My major research focus was the development of a new Monte Carlo transport model ({\tt Lido}) for heavy-flavor propagation inside a quark-gluon plasma. It was originally proposed in \cite{Ke:2018tsh} and was later significantly improved in \cite{Ke:2018jem}.
The original model used a hybrid linearized-Boltzmann-Langevin approach. Perturbative scatterings are included in the Boltzmann transport part. 
In particular, the gluon absorption process is implemented to balance the medium induced radiation. 
Non-perturbative contribution to heavy quark dynamics is modeled by a Langevin equation.
The heavy quark transport coefficient $\hat{q}$ was extracted using this model by comparing to experimental data in a global Bayesian analysis (Figure \ref{fig:qhat_Ds}).
\begin{figure}[ht]
\centering
\begin{minipage}{.48\textwidth}
  \centering
  \includegraphics[width=\textwidth]{qhat_p_T.png}
  \caption{The 95\% credible region of $\hat{q}$ extracted from data using the original {\tt Lido} model.}\label{fig:qhat_Ds}
\end{minipage}
\hfill
\begin{minipage}{.48\textwidth}
  \centering
  \includegraphics[width=\textwidth]{spectrum_L.png}
  \caption{Compare simulated radiaiton spectrum to theory calculation \cite{CaronHuot:2010bp} in a finite medium.}\label{fig:spectra}
\end{minipage}
\end{figure}

Recently, the model is improved in terms of parametrization flexibility and physical accuracy. 
First a separation scale $Q_{\textrm{cut}}\propto T$ is introduced to the {\tt Lido} model. 
Perturbative matrix-elements are only applied to large-momenta transfer ($Q > Q_{\textrm{cut}}$) processes while small-momenta transfer ($Q < Q_{\textrm{cut}}$) processes are absorbed into the Langevin equation. 
Tuning the scale parameter $Q_{\textrm{cut}}$ interpolates the linearized Boltzmann equation to a Langevin equation smoothly.
Second, I improved the Monte-Carlo implementation of the Landau-Pomeranchuk-Migdal (LPM) effect.
It is worth mentioning that the resulting simulation quantitatively agrees with the theoretical calculations of gluon radiation spectra (Figure \ref{fig:spectra}).

Currently, I am working on a tentative modeling of the vacuum-like radiation. 
In contrary to a common practice in tranport simualtions that vacuum radiation remains unmodified, a recent study \cite{Caucal:2018dla} showed that vacuum-like radiation should be excluded in a certain phase-space occupied by the medium induced radiation.
I implemented this phase-space subtraction in a {\tt Pythia + Lido} simulation to study its phenomenological impacts.
The preliminary result (Figure \ref{fig:prelim}) shows a notable increase of high-$p_T$ $R_{AA}$ due to the subtraction while the momentum anisotropy $v_2$ is slightly modified.
So, the improved treatment of vacuum-like radiation helps to solve the problem of simultaneous description of $R_{AA}$ and $v_2$, in addition to other possible solutions.
With these improvements, I am also moving towards an updated global Bayesian analysis.

\begin{figure}[ht]
\includegraphics[width=.85\textwidth]{Lido_obs.png}
\caption{Preliminary results using $\alpha_s(Q\geq\pi T)$ compared to ALICE \cite{Acharya:2018hre} and CMS \cite{Sirunyan:2017plt} measurements. Solid lines: unmodified pythia vacuum shower followed by heavy quark energy loss. Dashed: use phase-space subtracted pythia shower instead.}\label{fig:prelim}
\end{figure}

Finally, I am also part of the {\tt JetScape} collaboration since 2017. 
It is an NSF funded collaboration aiming for the next generation event generators for jet and bulk physics \cite{JetScape}. 
I contribute to computational work-flow testing of bulk medium simulation and also support the development of the statistical package.
Another graduate student and I will be responsible for integrating the {\tt Lido} model to {\tt JetScape} package.

\section*{Future research direction}
My prime interest is the precision study of heavy flavor and jet transport theories. 
The heavy-ion experimental program upgrades are bringing the measurements into the precision era. 
To make the best out of future high-quality data, I would like to contribute to the development of next generation transport models and to the understanding of theoretical uncertainties.
It is an important task because model / theory uncertainties affect the interpretation of the data and therefore, the understanding of the jet transport.
My road-map is to first calibrate jet transport algorithm to leading order perturbative calculation and validate its physical accuracy. 
After this, a systematic study on the discerning power of proposed observables can be performed.
As a long-term goal, a transport model validated at leading order will be the starting point to consider next-to-leading order effects and dynamical medium corrections.
A first step would be a sensitivity study on the relative importance of various higher order corrections.
The theoretical uncertainty can be understood in such a practice to evaluate how reliable the perturbative description is for jet transport phenomenon.
To push this effort, it would be ideal to work closely with jet theory community so that my transport simulation and statistical analysis expertise can meet the frontier of theory developments. 

\section*{Research accomplishments}
\subsection*{Parametric 3D initial condition of heavy-ion collisions}
To constrain a three-dimensional fluctuating entropy production created at early times in heavy-ion collision events, I extended the boost-invariant parametric initial condition model T\raisebox{-.25em}{R}ENTo \cite{Moreland:2014oya} to include space-time rapidity dependence \cite{Ke:2016jrd}. 
The local rapidity distribution of the entropy production is parametrized by its first three cumulants. 
Each cumulant depends on the local nuclear participant densities in a parametric way, satisfying basic symmetry properties.
Couple the parametric initial condition to a 3+1D hydrodynamic simulation with a hadronic afterburner, the parametric dependence can be tuned to reproduce the charged particle pseudorapidity density and two-particle rapidity correlation measured at the LHC. 
The functional form of the local longitudinal entropy distribution is inferred in such a way and is shown to be rather robust against different choices of the parametrization.
A three-dimensional structure of the entropy production at the onset of the quark-gluon plasma phase is reverse engineered from experimental data for the first time.   


\subsection*{A linearized-Boltzmann-Langevin model ({\tt Lido}) for heavy-flavor transport}
The linearized Boltzmann equation and the Langevin equation are two popular models to study heavy quarks transport phenomena.
The Boltzmann equation assumes heavy quark scatterings with medium quasi-particles at weak coupling; while the Langevin equation, assuming frequent and soft momentum exchange, does not require the existence of quasi-particles.
I developed the {\tt Lido} model that interpolates between these two pictures to generalize the transport modeling \cite{Ke:2018tsh, Ke:2018jem}.
The key feature is a tunable separation scale $Q_{\textrm{cut}}\propto T$. Interactions between heavy quarks and the medium with large-momenta transfers ($Q > Q_{\textrm{cut}}$) are modeled by a linearized Boltzmann equation with perturbative QCD matrix-elements; while small-momentum transfer ($Q < Q_{\textrm{cut}}$) interactions and non-perturbative contribution are modeled by a Langevin equation. 
The perturbative matrix-elements are exclusively applied to large-momentum transfer processes so that the values of running coupling constant  are well controlled. 
Using an empirical transport coefficient in the Langevin equation grants the model a flexible parametrization for the soft and non-perturbative contribution. 
Its functional forms can be extracted in a systematic model-to-data comparison.

\subsection*{Global Bayesian analysis in the extraction of heavy-quark transport property}
It is extremely difficult to calculate transport coefficients from first principle. 
An alternative way is to determine the probability distribution of physical parameters, such as heavy quark $\hat{q}$, from a global Bayesian model-to-data comparison.
Based on the Bayes theorem, the distribution of the true physical parameters breaks down to a product of a prior probability distribution and the likelihood function.
The ``prior" encodes the existing knowledge of a reasonable range to vary the parameters and the ``likelihood function" accesses the closeness between model calculations using certain parameters and the experimental data.
Such an analysis allows a global comparison to all available data to maximize the constraining power.
Moreover, instead of proposing a ``best" set of parameters, experimental and theoretical uncertainties are propagated to the distribution of model parameters.
Therefore, it is an ideal tool to quantify the amount of information one can learn from phenomenology.
I performed a global analysis by comparing calculations from the original {\tt Lido} model to the $R_{AA}$ and $v_{2}$ measurements at the LHC. 
The results favors a decreasing $\hat{q}/T^3$ with temperature.
The heavy quark spatial diffusion constant $D_s$ is comparable to the lattice calculations in the static limit.

\subsection*{Couple open heavy flavor evolution to quarkonium transport}
Quarkonia can dissociate in the medium and can also be regenerated by the recombination of adjacent heavy and anti-heavy quark pairs. 
Therefore, a consistent modeling of quarkonia production and transport in heavy ion collisions requires reliable knowledge on open-heavy flavor evolution.
In collaboration with a senior group member, we coupled the open heavy flavor transport model to the evolution of quarkonia inside the QGP \cite{Yao:2018zrg}.
It is found that the heavy quark diffusion is important for the thermalization of quarkonia.
Model parameters related to heavy quark transport were first tuned in the Bayesian analysis.
Then, parameters in the quarkonia sector are tuned to reach a global agreement with experimental measurements.

\subsection*{Modeling of quantum coherence in a transport approach}
At high-$p_T$, parton energy loss is dominated by gluon radiation. 
However, a Monte-Carlo implementation of gluon radiation in a dense medium is non-trivial, because it is a coherent process over a long distance  compared to the collision mean-free-path, known as the QCD LPM effect. 
Moreover, in a realistic collision event, the formation time can be comparable to the medium expansion time scale, medium fluctuation scale, and even the size of the QGP fireball. 
Therefore, the treatment of coherence phenomena in an event-by-event Monte-Carlo simulation is of both theory interest and phenomenological importance.
To mimic the long-distance coherence in a transport model, I treated the virtual system of the quark and daughter gluon as a ``quasi-particle" that interacts with a modified cross-sections until the gluon forms \cite{Ke:2018jem}. 
I also validated this approach by comparing the Monte-Carlo simulation to leading order theoretical calculations of gluon radiation spectra and energy loss, in both static / expanding, finite / infinite media. 
It was shown that simulations agree with theory within $\pm 15\%$. 
This level of accuracy is essential for a reliable extraction of jet transport properties from experimental measurements.

\bibliographystyle{apsrev4-1}
\bibliography{citation}

\end{document}
