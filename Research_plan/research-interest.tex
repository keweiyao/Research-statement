\documentclass[10pt,a4paper]{article}
\usepackage[a4paper,total={6in, 10in}]{geometry}
\usepackage[utf8]{inputenc}
\usepackage{amsmath}
\usepackage{amsfonts}
\usepackage{amssymb}
\usepackage{hyperref}

\usepackage{amsthm}
\usepackage{amscd}
\usepackage{graphicx}%
\usepackage{fancyhdr}

\textwidth6in

\setlength{\topmargin}{0in} \addtolength{\topmargin}{-\headheight}
\addtolength{\topmargin}{-\headsep}

\setlength{\oddsidemargin}{0in}

\oddsidemargin  0.0in \evensidemargin 0.0in \parindent0em

\pagestyle{fancy}
\lhead{Research Statement} 
\rhead{\today}
\chead{{\large{\bf Weiyao Ke}}} 
\lfoot{} 
\rfoot{\bf \thepage} 
\cfoot{}

\begin{document}

My research focus on the relativistic heavy-ion collisions phenomenology.
I conduct my graduate research at Duke University under the supervision of Professor Steffen Bass.

I have been developing a Monte-Carlo transport model for heavy-flavor that quantitatively agrees with calculations from thermal field theory.
Based on this model, my colleagues and I extracted the heavy quark transport coefficient from a Bayesian model-to-data comparison.
More recently, I start to study the Monte-Carlo implementation of the interplay between vacuum and medium-induced processes in the heavy-flavor context.
I come to understand that we need precision implementation of theory ideas to make the best use of future high-precision jet measurement.
I would like to continue my research in this direction and working closely within the jet theory group.

\section{Summary of graduate research}
I have been working with Professor Steffen Bass at Duke University since 2014. I summarized my major research projects as follows which I will described below,

\subsection{A parametric 3D initial condition for hydrodynamic simulations of heavy-ion collisions}
I extended the boost-invariant parametric initial condition model T\raisebox{-.25em}{R}ENTo to include space-time rapidity dependence, parameterized by the first three moments of the local rapidity distribution of the produced matter. Each moments depends on the local nuclear participant densities in a parametric way. Using a 3+1D hydrodynamic simulation with a hadronic afterburner, these parametric functions can be tuned to reproduce charged particle pseudorapdity density and two-particle rapidity correlation measured at the LHC. This way, we reverse engineered the local rapidity distribution of the entropy production at the onset of the quark-gluon plasma phase in the heavy-ion collision \cite{Ke:2016jrd}.

\subsection{The development of a linearized Boltzmann+Langevin model ({\tt Lido}) for heavy-flavor transport}

A phenomenological extraction of heavy quark transport property requires a flexible transport model with both theory inputs and tunable type of interactions. 
The {\tt Lido} model \cite{Ke:2018tsh, Ke:2018jem} was developed for this purposes. The key idea is that large momenta transfer scatterings of heavy quark with the medium is solved by a linearized Boltzmann equation with perturbative QCD matrix-elements while soft and frequent interaction with the medium is modeled by a Langevin equation. The transport coefficient of the soft part is extracted from a Bayesian model-to-data comparison.

\subsection{Extraction of heavy-flavor $\hat{q}$ using Bayesian statistics}

Currently, I am working an updated extraction of $\hat{q}$ by comparing the improved {\tt Lido} model to experimental data. 
The model has the advantages that its raditive process was implemented with controlled theory accuracy, and that the interaction model between hard parton and medium interpolates between pQCD matrix-element scattering approach and a radiation+diffusion approach. 
A systematic comparison to experimental measurements by coupling the model to the state-of-the-art bulk medium evolution model will results in an reliably extracted heavy quark transport coefficient with model assumption uncertainty folded in.

\subsection{An improved Monte-Carlo treatment of the Landau-Pomeranchuk-Migdal (LPM) effect}

The original {\tt Lido} model only implements the LPM coherence effect qualitatively. To improve the physical accuracy of the model, I developed an improved Monte-Carlo implementation of the LPM effect that can be tuned to agree with leading order theory calculation to $\pm 15\%$ \cite{Ke:2018jem}. It is very promising that with this improvement we can emulate leading order calculations in event-by-event evolving medium with quantified uncertainty.
This level of theory accuracy is essential to a reliable extraction of heavy-quark transport property when comparing to data. It is also a good basis for considering higher order effects in the future.


\subsection{The interplay between vacuum and medium-induced radiation in a Monte-Carlo modeling}

I am also working on an improved treatment of vacuum radiation in the transport approach. The transport approach is a formation that applies to partons with negligible virtuality, which is not true high$-p_T$ heavy quarks produced  in heavy-ion collision. 
We used to assume that vacuum radiation happens on a shorter time scale compare to that of the medium formation, and one can perform vacuum radiation followed by an in-medium transport at the onset of quark-gluon plasma.
This oversimplified picture is not true according to a recent study \cite{Caucal:2018dla}, where vacuum radiation may not only occupy the same space-time region as the medium-induced ones and but may also get modified by the medium.
We find this interplay between vacuum and medium-induced shower can be essential to understand of very high$-p_T$ heavy-flavor nuclear modification factor. 
We are working towards an interfacing scheme between Pythia vacuum shower and Lido in-medium transport to study the phenomenological implication of this effect.

\subsection{  Collaboration work: contribute to the JetScape collaboration}

I am also part of the JetScape collaboration since 2017. It is an NSF funded collaboration developing the next generation of event generators for both jet and bulk medium physics \cite{JetScape}. 
I contribute to the development and testing of computational work-flow for bulk medium simulation and also support the development of related statistical package for systematic model parameter calibration.
Another junior graduate student and I also focus on integrating the heavy-flavor transport model {\tt Lido} into the JetScape framework.

\section{Future plan}
The current situation is that theoretical calculations are often performed in idealized scenarios while phenomenology jet Monte-Carlo models, though include important qualitative features such as the Landau-Pomeranchunk-Midgal effect in different ways, lack quantitative comparison with known theory calculations.
This gap between the theory and Monte-Carlo tools can obscure the interpretation and understanding of the data.

This problem needs to be solved by designing a jet Monte-Carlo that quantitative agrees with theory in idealized limits and then apply to jets in heavy-ion collisions.

To push this effort, we will certainly benefit from a close connection with the jet theory community to discuss how to implement the jet physics more accurately in a Monte Carlo model and what novel effects can be studied in such a way. I have summarized my responsibility as follows,
\begin{itemize}
\item Study jet theory and familiarize with novel observables.
\item Improve the leading order implementation as it is  the baseline for studying new effects.
\item Try to interface vacuum and medium induced shower.
\item Benchmark the model predictions with uncertainties.
\end{itemize}

\bibliography{citation}
\bibliographystyle{ieeetr}
\end{document}
