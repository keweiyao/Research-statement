\documentclass[10pt,a4paper]{article}
\usepackage[a4paper,total={6in, 10in}]{geometry}
\usepackage[utf8]{inputenc}
\usepackage{amsmath}
\usepackage{amsfonts}
\usepackage{amssymb}
\usepackage{hyperref}

\title{Research Statement}
\author{Weiyao Ke \\
Email: \href{wk42@phy.duke.edu}{wk42@phy.duke.edu}\\
Advisor: Steffen A. Bass\\
Duke University, Durham NC
}
\date{}

\begin{document}
\maketitle
\section{Introduction}
My current research interest is the development of a tunable Monte-Carlo model that accurately reflects perturbative calculation, targeting the examination of theory assumptions using future precision measurement.

\section{Project description}
The understanding of jet and heavy-flavors in heavy-ion collisions requires good uncertainty control on both experimental and theory sides.
With measurements gaining precision, a reduction of prediction uncertainty is necessary.
The current situation is that theory calculations are often performed in idealized scenarios while phenomenology jet Monte-Carlo models, though include important qualitative features such as the Landau-Pomeranchunk-Midgal effect in different ways, lack quantitative comparison with known theory calculations.
This gap between the theory and Monte-Carlo tools can obscure the interpretation and understanding of the data.

This problem needs to be solved by designing a jet Monte-Carlo that quantitative agrees with theory calculations in idealized limits and then apply to jets in heavy-ion collisions.
I have made such an attempt in a recent paper \cite{}.
The Monte-Carlo simulated radiative energy loss and especially the radiation spectrum agrees with leading order calculation within $\pm 20\%$. 
This is very promising and it means that we can make leading order calculations in event-by-event evolving medium with a quantification of uncertainty.
Of course, this is only a first step towards precision prediction.
Other effects such as the interplay between the vacuum and the medium induced shower and possibly higher order corrections should also be implemented in a Monte-Carlo simulation to help understand the full picture of jets in heavy-ion collision.
Eventually, the predictions will be made by coupling the jet Monte-Carlo to a state-of-the-art medium evolution model which I have been familiarized with during my graduated study.

\section{Future plan}
To push this effort, we will certainly benefit from a close connection with the jet theory community to discuss how to implement the jet physics more accurately in a Monte Carlo model and what novel effects can be studied in such a way. I have summarized my responsibility as follows,
\begin{itemize}
\item Study jet theory and familiarize with novel observables.
\item Improve the leading order implementation as it is  the baseline for studying new effects.
\item Try to interface vacuum and medium induced shower.
\item Benchmark the model predictions with uncertainties.
\end{itemize}


\end{document}
